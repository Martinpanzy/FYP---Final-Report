\chapter{User Guide}
%put in here maybe pseudocode, or configuration file specifications etc.
This guide aims to support anyone who would like to use the delivered system or continue this project. Before launching this project, following packages must be installed: \texttt{ROS Kinetic}, \texttt{openni}, \texttt{zed-ros-wrapper}, \texttt{python-pcl}, \texttt{darknet\_ros}, \texttt{YuMi-ros-wrapper}, and \texttt{MoveIt!}. Some packages have their own dependencies. The detailed installation instructions can be found in the project Github repository \url{https://github.com/Martinpanzy/FYP-Yumi}.

The pose relationships between the two cameras and YuMi are defined in the YuMi launch files. The \texttt{demo.launch} is for simulation in Rviz, \texttt{yumi.launch} is for launching the real robot. 

For ASUS Xtion camera, the current relationship is defined as:
\begin{minted}[frame=single, framesep=1.5mm, baselinestretch=1, fontsize=\footnotesize, linenos, breaklines]{xml}
<node pkg="tf" type="static_transform_publisher" name="rgb_tf" required="true" args="0.2013 0.0641 0.6934 -0.0238 0.4836 -0.0141 0.875 /yumi_base_link /camera_link 100"/>
\end{minted}

For ASUS Xtion camera, the current relationship is defined as:
\begin{minted}[frame=single, framesep=1.5mm, baselinestretch=1, fontsize=\footnotesize, linenos, breaklines]{xml}
<node pkg="tf" type="static_transform_publisher" name="rgb_tf" required="true" args="0.2263 0.0541 0.7134 -0.0238 0.4836 -0.0141 0.875 /yumi_base_link /zed_left_camera_frame 100"/>
\end{minted}

The resolution and other settings of ZED Mini camera can be modified in the \texttt{common.yaml} file in the \texttt{zed-ros-wrapper} package. For YOLO detection, different versions can be used by modifying the \texttt{yolo.yaml} file in the \texttt{config} folder of \texttt{darknet\_ros} package and adding corresponding \texttt{.cfg} and \texttt{.weights} files to \texttt{yolo\_network\_config} folder. For more detail, please see the Github repository.

The computer vision and motion planning implementation methods introduced in this report are included in folder \texttt{shoehole} and \texttt{shoelace} respectively. Within these two folders:

\begin{itemize}
    \item \texttt{shoe\_asus.py:} provides functions of shoe detection, calculation of required locations for shoe pose adjustment, 6D shoe hole pose estimation etc using ASUS Xtion camera.
    \item \texttt{shoe\_zed.py:} provides same functionalities as shoe\_asus.py except it is for ZED Mini camera.
    
    \item \texttt{go\_asus.py:} provides shoe pose adjustment, shoelace insertion, grabbing, and pulling, as well as offset adjustment functionalities while using ASUS Xtion camera.
    \item \texttt{go\_zed.py:} provides same function as go\_asus.py except it is for ZED Mini camera.
    \item \texttt{yumi\_moveit\_utils.py:} includes several functions which are used in go\_asus.py and go\_zed.py.
\end{itemize}

To launch this project, YuMi needs to be set up firstly. To do this, please follow these steps:
\begin{enumerate}
\item Turn on YuMi through the power switch.
\item Connect the ethernet cable on XP23 port to your machine.
\item Turn on motors on controller interface (toggle physical button with 3 horizontal lines on the FlexPendant).
\item Switch to auto mode (toggle physical button with 2 horizontal lines on the FlexPendant).
\item Point to main programs (toggle physical button with 1 horizontal line on the FlexPendant).
\item Press the play button, both YuMi's grippers should then calibrate.
\end{enumerate}

Once YuMi has been setup, the shoe can be placed on the workbench. Whne using ZED Mini camera, following command should be run from a terminal after \texttt{roscore}:
\begin{enumerate}
\item \texttt{roslaunch zed\_wrapper zedm.launch}: to launch ZED Mini camera
\item \texttt{roslaunch darknet\_ros darknet\_ros.launch}: to launch YOLO detection
\item \texttt{roslaunch yumi\_shoe yumi.launch}: to launch YuMi ROS Nodes
\item \texttt{rosrun yumi\_shoe shoe\_zed.py}: to launch computer vision module
\item \texttt{rosrun yumi\_shoe go\_zed.py}: to launch motion planning module
\end{enumerate}

YuMi will then starts to manipulate the shoe or shoelace depends on the pose of the shoe. For more information, please see the project Github repository.