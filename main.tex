\documentclass[12pt]{report}

\usepackage{subfigure}
\usepackage[english]{babel}
\usepackage[utf8x]{inputenc}
\usepackage[T1]{fontenc}

\usepackage{minted}
\usepackage{graphicx}
\usepackage{float}
\usepackage{pgfgantt}   
\usepackage{lipsum}
\usepackage{graphicx}
\usepackage{natbib}
\usepackage{setspace}
\PassOptionsToPackage{hyphens}{url}

%% Sets page size and margins
\usepackage[a4paper,top=3cm,bottom=3cm,left=3cm,right=3cm,marginparwidth=1.75cm]{geometry}

%% Useful packages
\usepackage{amsmath}
\usepackage{graphicx}
\usepackage[colorinlistoftodos]{todonotes}
\usepackage[colorlinks=true, allcolors=black]{hyperref}

\begin{document}
\begin{titlepage}
    % \newgeometry{top=25mm,bottom=25mm,left=38mm,right=32mm}
    \setlength{\parindent}{0pt}
    \setlength{\parskip}{0pt}
    % \fontfamily{phv}\selectfont
    
    {
                    \Large
                    \raggedright
                    Imperial College London\\[17pt]
                    Department of Electrical and Electronic Engineering\\[17pt]
                    Final Year Project Report 2019\\[17pt]
    
    }
    
    \rule{\columnwidth}{3pt}
    \vfill
    \centering
      \includegraphics[width=\columnwidth,height=80mm,keepaspectratio]{title/YuMicover.jpeg}
    \vfill
    \setlength{\tabcolsep}{0pt}
    
    \begin{tabular}{p{40mm}p{\dimexpr\columnwidth-40mm}}
                    Project Title: & \textbf{Robot Manipulation of Shoe Laces -- AI Planning and Control} \\[12pt]
                    Student: & \textbf{Zengyang Pan} \\[12pt]
                    CID: & \textbf{01046157} \\[12pt]
                    Course: & \textbf{4EM} \\[12pt]
                    Project Supervisor: & \textbf{Prof Yiannis Demiris} \\[12pt]
                    Second Marker: & \textbf{Dr Tae-Kyun Kim} \\
    \end{tabular}
\end{titlepage}

\onehalfspacing
\setlength{\parskip}{1em}
\begin{abstract}
Robot motion planning based on computer vision is an essential topic in the field of AI. While the majority applications allow for relatively high manipulation tolerance, such as pick and place. 

In this project, I address the precise AI planning and control problem of putting shoelaces on a marked shoe using a bimanual robot, ABB YuMi, and an external camera. Starting with one arm holding one end of the lace, YuMi can adjust the orientation of the shoe if necessary, then pass the shoelace through the shoe hole. As an extension, YuMi will also plan sequences of trajectories for completing more holes up to finishing the whole shoe.

The main emphasis of this project is on computing the precise 6D pose of the shoe hole to estimate the approaching posture of YuMi’s arm accordingly. The final approach to tackle this problem is by combining random sample consensus (RANSAC) technique and some pre-defined constraints. Other topics, including dual arm motion planning, object detection and localization, and accurate offset adjustment, are also covered. Every single function, as well as the integrated system, is then experimented and evaluated.
\end{abstract}

\renewcommand{\abstractname}{Acknowledgements}
% It is usual to thank those individuals who have provided particularly useful assistance, technical or otherwise, during your project. 
\begin{abstract}
I would like to firstly express my sincere gratitude to my supervisor Prof Yiannis Demiris for his helpful advice, constructive feedback, and continuous support throughout the course of my project.

Furthermore, I would like to thank all the members in the Personal Robotics Lab, in particular, Urbano Miguel Nunes and Rodrigo Chacon Quesada, for their practical tips and friendliness, which kept me going whenever I was faced with a problem.

Also, I would like to thank my friends around the world, who always bring me happiness and encouragement as well as the wine we drank, the scenery we saw and the things we went through together.

Finally, I would like to thank my family, for their unconditional support since I was born and giving me the opportunity to study at Imperial College.
%Thanks mum, dad, my family!

%Thanks my supervisor, friends, PhDs in Lab and YuMi!

%Thanks breakfast, lunch, and dinner!

%Thanks gym, sunshine, coffee, movies, and London!

%Finally, thanks my brain!
\end{abstract}


\setlength{\parskip}{0em}
\tableofcontents
\newpage

%\listoffigures
%\listoftables
\setlength{\parskip}{0.7em}
\chapter{Introduction}
%It should begin with a clear statement of what the project is about so that the nature and scope of the project can be understood by a lay reader. 

In the past decades, robots have become an indispensable role in human life. There are various types of robots in the fields of manufacturing, healthcare, education, entertainment and military etc. No matter which category a robot belongs to, it is essential to endow the robot with some abilities so that it can perform corresponding tasks. Currently, there are mainly three approaches to teach skills to robots: direct programming, imitation learning and reinforcement learning, according to \citep{Kormushev_MDPI_2013}. However, there exists a trade-off between the difficulty to teach and computational complexity. Thus, a different approach should be adopted based on task specification.

\section{Motivation}
Speaking of non-industry environments, robot manipulation tasks usually cannot rely on hard-coded knowledge about the scene structure. This is because the environment can be modified by human actions, which is unforeseeable. Thus, a vision-based object recognition and localization system is extremely useful for providing the robot with scene knowledge. By combining computer vision techniques with motion planning, it enables robots to perform tasks more flexible and equip with a certain degree of autonomy.

\section{Objectives}
%ability independently to formulate and solve technical problems in project work. 
%project is intended to deliver, hardware, software, simulation, and analytical work.
Therefore, as my final year project, I aim to develop an algorithm for a bi-manual robot so that it can put shoelace on a shoe. In the core project, starting with one arm holding one end of the shoelace, the robot should detect a shoe hole and plan an arm trajectory to pass the shoelace through that hole. As an extension, the robot should be able to plan sequences of trajectories in order to complete the whole shoe.

\section{Challenges}
Due to the relatively small size of the shoe hole, this project requires a high degree of precision. Firstly, the optimal methods to localize the shoe hole need to be figured out. Then, the main challenge is to calculate the direction in which the lace should pass through the shoe hole in the dynamic environment. In addition, how to use robot arms to achieve this task accordingly also requires an accurate analysis. Finally, plenty of tests regarding the whole system need to be done in order to optimize the performance.

\section{Report Structure}
This report is structured as follows:

\begin{itemize}
    \item \textbf{Chapter 2 - Background} provides a background analysis of this project, including the robot and camera's choice, possible implementation approaches for both motion planning and computer vision side, as well as the chosen operating system and software. 
    \item \textbf{Chapter 3 - Requirement Capture} describes the detailed deliverables of the core project and extension, so that the reader can keep the objectives in mind while reading the remaining part.
    \item \textbf{Chapter 4 - Analysis and Design} introduces the designed system overview and workflow in order to achieve both the core project and extension. A brief introduction to the operating system and the robot status is included as well.
    \item \textbf{Chapter 5 and 6 - Implementation} gives the detailed approaches to fulfill the project requirements.
    \item \textbf{Chapter 7 - Testing and Results} presents a series of experimental results of the implemented methods. Success rate and execution time are the two evaluation criteria.
    \item \textbf{Chapter 8 - Conclusion} summarises the achievements of this project together with the advantages and limits. 
    \item \textbf{Chapter 9 - Future Work} suggests a few extra implementation ideas that could extend the scope of this project. 
    \item \textbf{Chapter 10 - User Guide} offers the instructions to launch this project.
\end{itemize}

\chapter{Background}
%Thorough discovery of background material. Decisions made in the project are correctly informed by background. Analysis of competing products, software standards, necessary software tools, background theory.
% The background section of the report should set the project into context by relating it to existing published work which you read at the start of the project when your approach and methods were being considered. The published work may be in the form of research papers, articles, text books, technical manuals, or even existing software or hardware of which you have had hands-on experience.

%link (and make explicit this link between) the different topics in a coherent pipeline: hardware-(choice of robot) > motion planning->state information (camera) (object detection +  orientation detection)-> etc
This chapter contains the background information of this project. Several potential implementation approaches are discussed and the design choices are made. As illustrated in Figure \ref{bgov}, the content starts with robot selection, then dives into motion planning algorithms. The available cameras that provide state information for the robot are then evaluated and chosen. After that, the techniques for objects 6D pose calculation are summarised and compared, which consists of object detection and orientation estimation. Finally, the system and software that used to implement the selected methods are introduced.

\begin{figure}[H]
\centering
\includegraphics[width = \columnwidth]{background/backgroundov.png}
\caption{The order of background introduction}
\label{bgov}
\end{figure}

\section{Choice of Robot}
There are three robots available for this project: iCub, Baxter and ABB YuMi.

The \citep{iCub} is a 1-metre tall humanoid robot for research in human cognition and artificial intelligence, which has 53 actuated degrees of freedom across its body. Moreover, it has facial expressions and can establish eye contact with users. However, its joints are relatively weaker than Baxter and YuMi. Considering this project only needs vision and robot arms, iCub is not considered as the best robot.

\begin{figure}[H]
\centering
\subfigure[iCub]{\includegraphics[width = 0.32\columnwidth]{background/icub.png}}
\subfigure[Baxter]{\includegraphics[width = 0.32\columnwidth]{background/baxter.png}}
\subfigure[ABB YuMi]{\includegraphics[width = 0.32\columnwidth]{background/YuMi.png}}
\caption{Three available robots}
\label{robot}
\end{figure}

The \citep{baxter} is a two-armed robot which is 3 feet tall and has a pedestal between 5'10" - 6'3" tall. Each of its arms has 7 degrees of freedom, whose movements are more reliable quicker than iCub. Baxter is usually used for simple industry jobs such as unloading, handing of materials. Considering its size, it is not suitable for this project. 

Referring to \citep{ABBsDual5:online}, YuMi is "a collaborative, dual arm, small parts assembly robot that includes flexible hands, parts feeding systems, camera-based part location and state-of-the-art robot control". With 7 degrees of freedom in each arm, it is designed for precise vision-based motion planning and thus is most appropriate to this project.

\section{Motion Planning}
The purpose of robotic motion planning is to seek the solution of the problem of "Go from the start configuration to the goal configuration while respecting all of the robot's constraints", such as avoiding collision with obstacles. A motion planning algorithm takes these tasks description as input, then calculates a path in configuration space. Below are several key concepts in robot motion planning problem.

\begin{itemize}
    \item \textbf{Work Space:} The physical space where the robot operates.
    \item \textbf{Configuration/State Space:} The configuration that describes the robot pose. For a solid 3D robot like YuMi, it can perform both 3D translation and 3D rotation. Therefore, a configuration needs 6 parameters: (x, y, z) for translation and (roll, pitch, yaw) for rotation, whose relationships are shown in Figure \ref{rpy}.
\end{itemize}

\begin{figure}[H]
\centering
\includegraphics[width = 0.4\columnwidth]{background/rpy.png}
\caption{Roll, pitch, yaw rotation related to x, y, z axis}
\label{rpy}
\end{figure}

\begin{itemize}
    \item \textbf{Obstacle Space:} The space that robot cannot move to. 
    \item \textbf{Free Space:} The set of configurations of robot that avoids collision with obstacles. In order to test if a configuration is in free space, forward kinematics will firstly be used to calculate the position of robot's geometry, then collision detection will be performed to check if it is collides with environment's geometry.
    \item \textbf{Target Space:} A subspace of free space that consists interested positions we would like the robot to move to. In this project, target space is determined by the camera.
    \item \textbf{Danger Space:} Space that robot is not desired but can move to. The robot usually pass through this space if the trajectory cannot be completed through free space.
\end{itemize}

According to \citep{OMPLPrim20:online}, there are plenty of methods still applicable in many scenarios, including exact and approximate cell decomposition, control-based methods, potential fields, and randomized planning. The summary of each method is as follow.

\begin{figure}[H]
\centering
\includegraphics[width = \columnwidth]{background/cell.png}
\caption{2D cell decomposition example where the blue regions are obstacles \citep{celldecom}. From left to right: approximate cell decomposition, exact cell decomposition, and motion planning using exact cell decomposition. The available path is through C3, C6, C9, C10, C12, C13, and C15.}
\label{cell}
\end{figure}

\begin{itemize}
    \item \textbf{Exact and Approximate Cell Decomposition:} This approach partitions the work space into discrete cells that corresponding to free space, where edges represent adjacency among cells and vertices indicate the individual cells (see Figure \ref{cell}). Therefore, the motion planning problem becomes a search from starting cell to the cell with goal position. This method has a trade off when setting the grid resolution. Coarser grid gives a faster search, but cannot find path for narrow parts of free space. Moreover, the number of points on the grid is exponentially proportional to configuration space dimension. This project has 6D state space, thus it is inappropriate to use this method.

    \item \textbf{Control-based Methods:} This method employed control theory and operates in continuous space. It typically uses feedback loops to manipulate the robot with minimal error. However, in complex dynamics environments, it is hard to calculate a feasible trajectory and the valid motions will be restricted.

    \item \textbf{Potential Fields:} This technique computes a vector at each point in the work space by summing up the an attractive force emanating from the goal, and a repulsive force from all obstacles and the boundaries (see Figure \ref{potentialexample}). It treats the configuration of robot as a point in a potential field. Although it only requires little computation, navigating using gradient descent to follow the potentials to goal can fail due to local minima in the field. This issue occurs when the attractive forces and repelling forces cancel each other out at a specific point.
\end{itemize}

\begin{figure}[H]
\centering
\subfigure[Potential field in a 2D space contains a goal and an obstacle \citep{potential}]{\includegraphics[height=5.3cm,keepaspectratio]{background/potentialfield.png}}
\subfigure[Potential field in 3D space \citep{3dpotential}. Warm colors represent high potential (obstacles) and cool colors indicate low potential (goals).]{\includegraphics[height=5.3cm,keepaspectratio]{background/potentialfield3d.jpg}}
\caption{Example of the potential field approach}
\label{potentialexample}
\end{figure}

\begin{itemize}
    \item \textbf{Randomized Planning:} Randomization can be effective in motion planning. For example, in potential fields, it has been shown that Brownian motions where applying a random action for a specific amount of time can effectively guide a system out of the local minima.
\end{itemize}

The state-of-art sampling-based motion planning was inspired by randomization \citep{OMPLPrim20:online}, which employs sampling of the configuration space of the robot to effectively answer planning queries. It is especially suitable for systems with many degrees of freedom or differential constraints. Due to motion constraints and size of configuration space, traditional methods can spend a long time to address these problems. While sampling-based planning reasons over a finite set of configurations and connects these samples through collision free paths. Usually, this method gives a probability to complete, as the the number of samples reasoned increases, the probability will converge to 1 if a solution exists. Nevertheless, it cannot determine if a problem has no solution, which can only be speculated by low probability. 

There are many existing sampling-based algorithms, such as Rapidly-expanding Random Trees \citep{RRT}, Probabilistic Roadmap Method \citep{Kavraki1996ProbabilisticRF}, Kinodynamic Planning by Interior-Exterior Cell Exploration \citep{Sucan2012AST} etc. Two typical planners are discussed below.

\begin{figure}[H]
\centering
\subfigure[The 2D work space and a single robot state]{\includegraphics[width = 0.45\columnwidth]{background/rm1.png}}
\subfigure[One possible set of uniform random samples of free space]{\includegraphics[width = 0.45\columnwidth]{background/rm2.png}}
\subfigure[PRM connects samples that are close to one another using a straight path in the free space]{\includegraphics[width = 0.45\columnwidth]{background/rm3.png}}
\subfigure[The start and goal are connected to the roadmap, shortest path is found]{\includegraphics[width = 0.45\columnwidth]{background/rm4.png}}
\caption{Probabilistic roadmap example \citep{OMPLPrim20:online}}
\label{prm}
\end{figure}

\begin{itemize}
    \item \textbf{Probabilistic Roadmap (PRM)} \citep{Kavraki1996ProbabilisticRF}: The idea of this approach is to use random sampling of configuration space to construct a roadmap of free space. Figure \ref{prm} illustrates a simple example of PRM, where a 2D work space and a point robot are used.
    
    The shaded areas represent obstacles. In sampling-based algorithms, free space usually cannot be known. As shown in Figure \ref{prm} (b), after perform collision check, the collision free samples are kept. Once sufficient free samples have been discovered, the roadmap will be built by connecting these samples. Utilizing a local planner whose task is to find short collision free path, each sample will be connected to its k nearest neighbors. Figure \ref{prm} (c) is the complete roadmap for this example. Finally, the start and goal states will be added to the roadmap and a graph search can be fulfilled to compute the shortest path.
    
    \item \textbf{Tree-based Planners:} There are various types of tree-based planners (e.g. \citep{RRT}, \citep{Sucan2012AST}, \citep{inproceedings}, and \citep{hsu}). Here I will describe a general framework, as shown in Figure \ref{tree}. Notice that the biggest difference of tree-based techniques and PRM is that the former structure has no cycles.
    
    Starting from the start state, tree-based planners use random sampling to expansively explore next samples along a collision free path. Figure \ref{tree} (a) illustrates an example of first few valid samples. These planners usually bias their expansion towards the goal, if the existing tree is able to connect to the goal, then they complete the search. Figure \ref{tree} (b) is an example that the goal state cannot be directly connected to the tree, while \ref{tree} (c) shows the case where it can.
    
    Finally, because tree-based planners are directed and acyclic, they are more appropriate for single-query planning and complex dynamics.
\end{itemize}

\begin{figure}[H]
\centering
\subfigure[The first few random valid samples have been connected to the tree using an expansion heuristic]{\includegraphics[width = 0.45\columnwidth]{background/rm5.png}}
\subfigure[Case when the goal cannot be connected to the tree because of an obstacle]{\includegraphics[width = 0.45\columnwidth]{background/rm6.png}}
\subfigure[Case when the goal can be connected to the tree]{\includegraphics[width = 0.45\columnwidth]{background/rm7.png}}
\caption{Tree-based planners example \citep{OMPLPrim20:online}}
\label{tree}
\end{figure}

The memory requirement of sampling-based techniques is lower than other types of planners, because no explicit representation of free configuration space is required. It is extremely powerful to plan in complex dynamics and high-dimensional space. However, implementing these algorithms in a generic way is non-trivial. Thus, Open Motion Planning Library (OMPL) \citep{OMPL} is employed to plan the YuMi arm trajectories in this project, which is also available for use through the Robot Operating System (ROS).
In this case, after defining the configuration space, only start state, goal state, and environment scene need to be supplied to plan a single motion. 

\section{Choice of Camera}

Due to the fact that this project requires 3D localization of interested objects (shoes, shoe hole etc.), the camera must be able to provide depth information, which allows user to segment images. There are two types of RGB-D cameras available in the Lab, ASUS Xtion PRO and Microsoft Kinect. Both cameras provide similar functionalities, they all have an infrared emitter (leftmost), RGB camera (middle), and an infrared depth camera (rightmost). Their resolutions are also the same, which is $640 \times 480$. Considering the ASUS Xtion PRO has already been fixed on the robot, I used it directly for this project.

\begin{figure}[H]
\centering
\subfigure[ASUS Xtion PRO]{\includegraphics[width = 0.35\columnwidth]{background/asus.jpg}}
\subfigure[Microsoft Kinect]{\includegraphics[width = 0.3\columnwidth]{background/kinect.png}}
\subfigure[ZED Mini]{\includegraphics[width = 0.3\columnwidth]{background/zedmini.jpg}}
\caption{Three available cameras}
\label{camera}
\end{figure}

In the later stage of this project, another stereo camera named ZED Mini became available. It has two lenses with a separate film frame for each lens, which allows it to simulate human vision, and capture three-dimensional images. Its resolution is up to $2208 \times 1242$, much higher than that of ASUS Xtion and Kinect. Therefore, I ended up choosing this camera.


\section{Object Detection} \label{od}
Object detection is considered as one main part of computer vision, which can provide 2D positions of interested objects from an image. With the help of corresponding depth information provided by ASUS Xtion or ZED Mini, 3D location can then be determined. At present, there are several existing methods to detect the object's 2D location.

Due to the fact that the number of occurrences of interested objects is not fixed, and they might have different spatial locations within the image and different aspect ratios, taking different regions from the image and using CNN to classify the presence of the object could computationally blow up \citep{RCNNFast65:online}. Therefore, plenty of detection algorithms like R-CNN \citep{RCNN}, YOLO \citep{YOLO} etc have been developed to tackle this problem. 

The standard R-CNN method is not a complete end-to-end object detector and painfully slow. Fast R-CNN \citep{FastRCNN} improved the detection accuracy and speed, but the model relies on an external region proposal algorithm. With the propose of Faster R-CNN \citep{FasterRCNN}, R-CNNs became an end-to-end object detector which relies on a Region Proposal Network and removes the Selective Search requirement. The advantage of R-CNNs is their accuracy, but the biggest problem of them is their speed, obtaining only 5FPS on a CPU, which is unsuitable for real-time object detection. YOLO employs a one-stage detector strategy to increase the speed of deep learning based object detectors, which obtains 45FPS on a GPU. According to \citep{YOLO9000}, YOLO9000, which trained on both COCO detection dataset and ImageNet classification dataset, is capable of detecting over 9000 separate classes. Although it seems to be suitable for this project, shoelaces and shoe holes are not included in its detection range and its accuracy needs to be tested. YOLOv3 \citep{YOLOv3} is the latest update of YOLO family, however, it is trained only on the COCO dataset that consists 80 labels. The category of shoes is not included, thus it is not suitable for this project. 

Another approach of detecting an object is color detection, which is available for this project. The reason is that shoes with special laces and holes color will be selected as the manipulated object, which should be easily distinguished from background. \citep{cie} converted RGB image to CIE-Lab color space to increase the accuracy of color segmentation. \citep{HSV} performed object segmentation using color filtering based on HSV (Hue, Saturation, Value) and calculated the centroid of the target. Although depth information can help discriminate objects that are not in the same plan as the object of interest, this method still has disadvantage of been sensitive to the changes on lighting and is better being applied on indoor environment. By carefully setting the RBG parameters, this approach will be considered to apply on shoe holes and laces detection.

Shape detection is reckoned as an alternative way, which is achieved by segmenting point clouds and is suitable for detecting shoes. Furthermore, \citep{deformable_track} introduced an algorithm for tracking deformable objects form a sequence of point clouds, which is able to robustly track one-dimensional object such as laces. However, the method is not ready for public consumption right now and following the mathematics steps in the paper to program is quite difficult. 

\citep{vision_touch} presented an object-tracking framework that combines point cloud information from an RGB-D camera with GelSight contact sensor's tactile information. It also improves the pose accuracy during contact and provides robustness to occlusions of small objects. However, it is not suitable for YuMi robot.

Furthermore, OpenPose \citep{openpose} can jointly detect key points of the human body, hand and foot etc with high accuracy and reliability from images, which can also be a backup plan for shoe localization. The only drawback of this approach is that people must wear the shoes in order for the algorithm to locate them, which will increase the complexity of experiments and letting robot manipulate the wearing shoes is considered to be more difficult.

According to these above research, this project will first employ YOLO9000 to detect shoes, and use OpenCV color detection to detect holes and laces. Based on their performances, other approaches will be considered whether to be applied or not.

\section{Orientation Estimation} \label{oriestimation}
In order to manipulate objects using robot, the 3D orientation of interested objects needs to be calculated besides its 3D location (discussed in Section \ref{od}). In this project, particularly, the orientation of shoe holes has to be computed so that the shoe lace can be successfully put into them.

There are several existing state-of-the-art 6D object pose estimation (3D location and 3D orientation) methods. \citep{singleshot} proposed a real-time single-shot approach to predict object's 6D pose, which does not require additional post-processing. They first employed CNN to compute 2D image locations of projected vertices of the object’s 3D bounding box. Then a PnP algorithm is utilized to estimated 6D pose. However, these kind of methods become unreliable when low-texture or low-resolution inputs are given. \citep{PoseCNN} introduced PoseCNN, estimates the object's 3D rotation by regressing to a quaternion representation using only color images. They also presented a novel loss function to let PoseCNN handle symmetric objects. Depth data is used for further refinement. \citep{DeepIM} presented DeepIM, given an initial pose estimation, it can refine the object's pose by matching the rendered image against the observed image iteratively. DeepIM only uses color images to introduce the untangled pose representation, which is independent of the 3D object model's coordinate frame and the object size. \citep{DenseFusion} proposed DenseFusion, which processes both color and depth information separately and employs a dense fusion network to extract pixel-wise dense feature embedding, from which the pose of known objects can be predicted. Moreover, it is robust against occlusions and outperforms PoseCNN after refinement on two benchmarks for 6D pose estimation, YCB-Video \citep{PoseCNN} and LineMOD \citep{linemod} respectively.

However, these algorithms only compute object (shoe) pose, after retraining them using shoe dataset which needs to be prepared by myself. In addition, they cannot calculate the orientation of specific region on the interested object, such as shoe holes'. In order to achieve this function, the 6D pose correlations of shoe holes and centroid of shoe need to be determined. After that, once these algorithms compute shoe pose, the pose of holes can be determined accordingly. Nevertheless, calculating these correlations are difficult and time consuming. Therefore, based on these considerations, those 6D pose estimation approaches cannot be my first choice.

To tackle this problem, another method I came up with is inspired by Random Sample Consensus (RANSAC) \citep{rsc}. RANSAC can estimate parameters of mathematical model from a set of data containing both inliers and outliers. It selects random subset of the original data interatively to form hypothetical inliers and then tests this hypothesis as follows:

\begin{enumerate}
    \item Fit a model to the hypothetical inliers.
    \item Test the fitted model using all other data. If there is a good fit between a data and the estimated model, then this data will be regarded as a hypothetical inlier.
    \item If there are sufficient data be considered as hypothetical inliers, the computed model is good.
    \item Re-estimate the model using all hypothetical inliers.
    \item Evaluate the model by estimating the error of the model related to the inliers.
\end{enumerate}

Each time of this procedure will produce a refined model with a corresponding error or a model that is rejected because only few data are classified as inliers. In the former case, the refined model will be kept if its error is lower than the last saved model.

\begin{figure}[H]
\centering
\subfigure[A dataset containing both inliers and outliers]{\includegraphics[width = 0.45\columnwidth]{background/rsc1.png}}
\subfigure[Fitted line with RANSAC, blue represents inliers, red repsents outliers]{\includegraphics[width = 0.45\columnwidth]{background/rsc2.png}}
\caption{RANSAC example \citep{random}}
\label{ransac}
\end{figure}

RANSAC can estimate the model parameters accurately even if the number of outliers is large. Its computation time is determined by the number of iterations and thus there is a trade-off between time and accuracy. Furthermore, the problem-specific thresholds need to be set as well. Figure \ref{ransac} is the example of applying RANSAC on 2D dataset to estimate a best fitting line.

For this project, the orientation of a shoe hole can be represented as the normal vector of it. The method I introduce is by applying RANSAC technique to construct a plane of interested shoe hole, then computing the normal vector accordingly. Moreover, with the help of some pre-defined constraints and data processing techniques, the accuracy and robustness of resulting orientation can be improved.

\section{System and Software}
All programs of this project will run under Ubuntu system. ROS will act as the middleware between YuMi and the software, which creates a network of Nodes in order to control different functionalities of the robot. By sending messages from one Node to others, communication can be achieved between Nodes. To be specific, the sender advertises on a given topic, while the receiver subscribes the corresponding topic. After compiling a standalone program into a ROS package, it is able to run on the ROS network. In addition, these programs need to be written in Python or C++.

Apart from that, MoveIt! will be used to provide motion planning functionality in ROS, which can load robots' URDF file and create appropriate state spaces for user-defined joint groups and call OMPL planners to find feasible paths. The path produced by OMPL are then translated by MoveIt! into dynamically feasible trajectories. Self-collisions can also be discovered on a pre-processing phase by MoveIt!.

%A file format for specifying articulated mechanisms is also provided by ROS, when parsed, cause the corresponding state space for planning to be created. 
\chapter{Requirements Capture}
%ability independently to formulate and solve technical problems in project work. 
%The project specification should state clearly what the project is intended to deliver, including all hardware, software, simulation.

\section{The Project Deliverable}
The objective of this project is to solve of problem of putting shoe laces on a marked shoe using bi-manual robot YuMi. Starting with one arm holding the shoelace, YuMi should detect a shoe hole and pass the shoelace through it by planning an arm trajectory. For the project extension, YuMi will also plan sequences of trajectories to complete more holes up to the whole shoe.

\begin{figure}[H]
\centering
\includegraphics[width = 0.5\columnwidth]{RequirementsCap/shoe.jpg}
\caption{The manipulated shoe with distinct lace and hole colors}
\label{shoe}
\end{figure}

As illustrated in Figure \ref{shoe}, the marked shoe for manipulation has distinct colors, with orange heads of shoelace and blue shoe holes. In addition, the entire manipulation process will be carried out on the workbench of the Imperial College Personal Robotics Lab under normal lighting condition.

The project deliverables contain both core project and extensions can be divided into two main parts: Computer Vision and Motion Planning.

\section{Computer Vision}
The main aim of this part is to compute the real-time 6D pose of a specific shoe hole. There are also other tasks include providing the YuMi end-effector with the positional information needed for shoe pose adjustment, calculating the shoelace orientation before entering the second hole, etc. The algorithm must provide an accurate and stable result so that it can be used for actual shoe lace manipulation. The following features should be delivered:

\begin{itemize}
    \item \textbf{Camera Setup (core project):} Setting up the working environment for two cameras and YuMi, measuring the pose relationships between them, and installing camera dependency packages.
    \item \textbf{Shoe Detection:} Detecting the 2D bounding box of the shoe when it is placing on the workbench.
    \item \textbf{Required Locations for Shoe Pose Adjustment (extension):} Computing a series of 3D locations required by YuMi in order for it to adjust the pose of the shoe.
    \item \textbf{3D Location of Shoe Hole (core project):} Computing the 3D real-world location of the centroid of target shoe hole.
    \item \textbf{3D Orientation of Shoe Hole (core project):} Computing the 3D orientation of that hole.
    \item \textbf{6D Pose of Shoelace after Pulling (extension):} Computing the 6D pose of the shoelace after pulling it out from the first shoe hole, in order to let another gripper align with and re-clamp it before inserting the next hole.
\end{itemize}


\section{Motion Planning}
This section focus on real-world motion planning of YuMi's arms. The outcome of this part should enable YuMi to adjust the orientation of the shoe if necessary, and then put shoe lace into a hole accurately while avoiding any collisions with obstacles. Finally, YuMi should pull the shoelace out to complete that hole and adjust the gripper pose to align with the shoelace and reclamp it. So that YuMi can start to plan trajectories for next shoe hole. These objectives can be divided into following tasks:

\begin{itemize}
    \item \textbf{MoveIt! Interface and Planning Scene Setup (core project):} Setting up the MoveIt! Python interface, and initializing the manipulating environment by defining known obstacles' dimension and position.
    \item \textbf{Movement Control (core project):} Setting up the control interface among my control instructions and YuMi's grippers and arms.
    \item \textbf{Safe Positions Calculation (core project):} Calculating several safe poses of YuMi's arms, including cal pose, home pose and initial pose.
    \item \textbf{Shoe Pose Adjustment (extension):} Adjusting the orientation of shoe using YuMi gripper if its pose is not ideal. 
    \item \textbf{Robot Gripper Approaching Pose (core project):} Computing the 6D approaching pose to the interested shoe hole, and defining a series of required waypoints.
    \item \textbf{Offset Adjustment and Shoelace Insertion (core project):} Adjusting the offset between camera readings and YuMi movement, especially offset introduced by the gripper's pose, and finally inserting the shoelace into the shoe hole.
    \item \textbf{Shoelace Grabbing (extension):} Computing the 6D shoelace grasping pose in order to pull it out, and a series of motions in preparation for calculating the 6D pose of shoelace.
\end{itemize}



%\section{Testing Environment}
%light pose robot .....
\chapter{Analysis and Design}

\section{ROS and YuMi Overview}
As discussed in the Background Chapter, ROS handles a number of independent Nodes that performs specific function(s). They communicate with each other by publishing messages to topics or subscribing them. ROS messages support several data formats, including number, string, and time etc. Moreover, these communications are only available if a $roscore$ is running, which is a collection of programs and Nodes that are prerequisites of a ROS-based system.

Besides, $TF$ is an important package in ROS and used several times in this project. According to \citep{tfROSWik}, it maintains the relationship between coordinate frames (such as camera frame, robot fram etc.) in a tree structure buffered, which enables the user to transform points, vectors, etc between any two coordinate frames at any desired point in time. Details will be discussed in Implementation Chapters.

The dual-arm robot YuMi communicates with ROS using Personal Robotics Lab setup (User Guide Chapter contains more details). YuMi has one gripper camera embedded in each of its gripper. Using these cameras may provide more accurate results of shoe and shoe hole locations. However, in order to use these two cameras, a package named $abb\_rws\_interface$ which is a proprietary interface of ABB is required. According to \citep{EGMfiles}, it is not public. Also, the open source version of that package called $abb\_librws$ cannot match the current $yumi\_cameras$ setup. Furthermore, it is not very feasible to tie a small camera to YuMi's arms. Therefore, this project will only utilize the ASUS Xtion or ZED Mini camera for detection.

\section{System Overview}

\begin{figure}[H]
\centering
\includegraphics[width = \columnwidth]{AnalysisDesign/system.png}
\caption{System overview}
\label{c4}
\end{figure}

Figure \ref{c4} displays the system overview of this project, which consists six components: the shoe, the camera, computer vision module, ROS Kinetic, motion planning module, and YuMi. Blue arrows represent computer vision messages and orange arrows indicate related motion plans and manipulations.

Starting from the camera, which consistently recording RGB and depth images of the workbench and outputs them to the computer vision module. The module then performs several functions including shoe detection, shoe hole tracking, 6D shoe hole pose estimation etc and publishes the corresponding message to ROS topics. After that, the motion planning module subscribes these information and plans YuMi arms trajectories accordingly to adjust shoe pose or put lace on a hole. If the planning succeeds, YuMi will then execute these motions to manipulate shoe and shoe lace.

%detail---------------------------
\section{System Workflow}

\begin{figure}[H]
\centering
\includegraphics[width = \columnwidth]{AnalysisDesign/workflow.png}
\caption{System workflow}
\label{workflow}
\end{figure}

discussion ................

The following two chapters will discuss detailed implementation approaches.
\chapter{Implementation - Computer Vision}

This chapter will cover the implementation methods on computer vision side of this project using either ASUS Xtion or ZED Mini. By adjusting some settings which will be discussed in this chapter, the algorithm works for both cameras. The entire code for this part can be seen in !!!!!!!!!!

\section{Camera Setup}
The dependency package allow specific camera to communicate with ROS is shown in Table \ref{camerapackage}, which contains corresponding launch files. Entering the launch command in the terminal can start the package driver. After that, a list of camera topics including RGB, depth, and point-clouds etc can be subscribed for further processing. 

\begin{table}[H]
\centering
\resizebox{\columnwidth}{!}{
\begin{tabular}{||c||c|c||}
\hline
Camera & Dependency package & Launch command \\ \hline \hline
ASUS Xtion & $openni2$ & $roslaunch$ $openni2\_launch$ $openni2.launch$ \\ \hline
ZED Mini & $zed-ros-wrapper$ & $roslaunch$ $zed\_wrapper$ $zedm.launch$ \\ \hline
\end{tabular}
}
\caption{The dependency packages and launch commands for Cameras}
\label{camerapackage}
\end{table}

In ROS, video streams are represented by these sequences of images. If user would like to use ROS images in conjunction with OpenCV, $CvBridge$ package provides the converting interface, which is used in this project as well.

\begin{figure}[H]
\centering
\includegraphics[width = \columnwidth]{Implementation/cv/yumicamera.jpg}
\caption{YuMi and ASUS Xtion camera position setup}
\label{5.1}
\end{figure}

Figure \ref{5.1} displays the relative position between YuMi and the two cameras. The cameras' location and orientation transform from YuMi frame $yumi\_base\_link$ are illustrated in Table \ref{cameraframeset} and written in the YuMi launch file (see !!!!!!!). These data are measured manually, which allow ROS $TF$ to perform frame transformation between any camera detected objects and YuMi in this project. 
\begin{table}[H]
\centering
\resizebox{\columnwidth}{!}{
\begin{tabular}{||c||c|c||}
\hline
Camera & Camera frame & Transform from $yumi\_base\_link$ \\ \hline\hline
ASUS Xtion & /camera\_link & {[}0.2013, 0.0641, 0.6934, -0.0238, 0.4836, -0.0141, 0.875{]} \\ \hline
ZED Mini & /zed\_left\_camera\_frame & {[}0.2263, 0.0541, 0.7134, -0.0238, 0.4836, -0.0141, 0.875{]} \\ \hline
\end{tabular}
}
\caption{Transform settings between YuMi and cameras}
\label{cameraframeset}
\end{table}

\section{Shoe Detection} \label{shoedetection}
As mentioned in Background Chapter, YOLO9000 will be used for shoe detection. \citep{bjelonicYolo2018} developed the YOLO ROS interface called $darknet\_ros$, which supports using YOLO on both CPU and GPU. This package depends on OpenCV, boost (C++ library) and CUDA (if use Nvidia GPU for faster processing). Detailed installation and setup instructions can be found in User Guide Chapter. 

\begin{figure}[H]
\centering
\includegraphics[width = 0.5\columnwidth]{Implementation/cv/yolofile.png}
\caption{Important files in YOLO package}
\label{yolofolder}
\end{figure}

Figure \ref{yolofolder} shows some important files in $darknet\_ros$. The $.weights$ file and $.cfg$ file contain the trained weights and labels etc for a specific YOLO version. Here, I use YOLO9000. The $yolo.yaml$ file points to these two files and defines a threshold value $0.3$. The results will be reported only if the YOLO9000 prediction probability of a detection class exceeds this value. 

The file named $ros.yaml$ defines the names and some parameters of the publishers, subscribers and actions of $darknet\_ros.launch$. Here, the input can be set as camera topic $/camera/rgb/image\_raw$ for ASUS Xtion, or $/zed\slash zed\_node\slash rgb\slash image\_rect\_color$ for ZED Mini, both contain the RGB images taken by the camera. It publishes several topics but only $/darknet\_ros/bounding\_boxes$ will be used in the following algorithms. The messages in this topic include the prediction class names of objects, their prediction probabilities and 2D bounding box coordinates ($xmin$, $xmax$, $ymin$, and $ymax$). If the category is predicted as 'shoe' or 'footwear', the corresponding box coordinates will be recorded. By speeding up with GPU, $darknet\_ros$ can update the detection results approximately every 0.05 seconds.

\begin{figure}[H]
\centering
\subfigure[RGB image]{\includegraphics[width = 0.45\columnwidth]{Implementation/cv/zedrgb.png}}
\subfigure[YOLO detection image including bounding boxes]{\includegraphics[width = 0.45\columnwidth]{Implementation/cv/zedyolo.png}}
\caption{ZED Mini camera image before and after using YOLO detection system}
\label{5.2zed}
\end{figure}

\begin{figure}[H]
\centering
\subfigure[RGB raw image]{\includegraphics[width = 0.45\columnwidth]{Implementation/cv/raw.png}}
\subfigure[YOLO detection image including bounding boxes]{\includegraphics[width = 0.45\columnwidth]{Implementation/cv/yolo.png}}
\caption{ASUS Xtion camera image before and after using YOLO detection system}
\label{5.2asus}
\end{figure}

Figure \ref{5.2zed} and \ref{5.2asus} are the examples of YOLO detection using different cameras.



\section{Required Locations for Shoe Pose Adjustment} \label{shoeadjust}

\begin{figure}[H]
\centering
\subfigure[]{\includegraphics[width = 0.15\columnwidth]{Implementation/cv/1.png}}
\subfigure[]{\includegraphics[width = 0.15\columnwidth]{Implementation/cv/2.png}}
\subfigure[]{\includegraphics[width = 0.15\columnwidth]{Implementation/cv/3.png} \label{ver1}}
\subfigure[]{\includegraphics[width = 0.15\columnwidth]{Implementation/cv/4.png} \label{ver2}}
\subfigure[]{\includegraphics[width = 0.15\columnwidth]{Implementation/cv/5.png}}
\subfigure[]{\includegraphics[width = 0.15\columnwidth]{Implementation/cv/6.png}}
\caption{Examples of possible shoe orientations}
\end{figure}

In the beginning, the shoe can be placed on the workbench in different directions, making the real manipulation difficult in some cases, especially when it is vertically placed (Figure \ref{ver1} and \ref{ver2}). Therefore, under this circumstance, its orientation need to be adjusted. Figure \ref{adjustmentidea} displays the idea. Starting at two blue points ($adl$ and $adr$), both of YuMi's grippers will move towards the orange points ($adll$ and $adrr$) simultaneously to perform the adjustment. After that, they will follow the previous paths and return to blue points in order to avoid potential collisions with the shoe. The ideal shoe pose after adjustment is shown on the right hand side.

\begin{figure}[H]
\centering
\subfigure[Before adjustment]{\includegraphics[width = 0.49\columnwidth]{Implementation/cv/adjustmentidea.png}}
\subfigure[After Adjustment]{\includegraphics[width = 0.49\columnwidth]{Implementation/cv/adjustmentidea2.png}}
\caption{The shoe pose adjustment idea}
\label{adjustmentidea}
\end{figure}

To achieve this, the 2D pixel coordinates of these blue and orange points will firstly be documented, which can be determined based on the YOLO bounding box. However, YuMi still needs to know the real-world 3D locations of these points in order to move its grippers to the correct places. This can be computed by looking at the camera registered point cloud topic, which is shown in Figure \ref{ptcloud}. For ASUS Xtion, the topic name is $/camera/depth\_registered/points$. While ZED Mini uses $/zed\slash zed\_node\slash point\_cloud\slash cloud\_registered$.

\begin{figure}[H]
\centering
\subfigure[RGB raw image]{\includegraphics[width = 0.45\columnwidth, height=50mm]{Implementation/cv/rgbpt.png}}
\subfigure[Depth registered point clouds]{\includegraphics[width = 0.45\columnwidth, height=50mm]{Implementation/cv/ptcloud.png} \label{ptcloud}}
\caption{Messages of different ASUS Xtion camera topics}
\end{figure}

Each point cloud contains the real-world $xyz$ readings relative to the camera and RGB information. Take ASUS Xtion for example, since its image resolution is $640 \times 480$, there are total $307200$ points for each image. Sometimes, the $xyz$ readings of some points might be $nan$, which are not reliable and should be ignored. Using those 2D pixel positions calculated previously, $pc2.read\_points$ function helps find the corresponding point cloud and return the reading.

\begin{minted}[frame=single, framesep=1.5mm, baselinestretch=1, fontsize=\footnotesize, linenos, breaklines]{python}
adr = list(pc2.read_points(data, field_names = ('x', 'y', 'z'), skip_nans = True, uvs = [(self.adr_x, self.adr_y)]))
if len(adr) > 0:
    adr_x, adr_y, adr_z = adr[0]
    adr = [adr_z, -adr_x, -adr_y] #if using ASUS Xtion
    #adr = [adr_x, adr_y, adr_z] #if using ZED Mini
    self._tfpub.sendTransform((adr), tf.transformations.quaternion_from_euler(0, 0, 0), rospy.Time.now(), "adr", 'camera_link')
    #self._tfpub.sendTransform((adr), tf.transformations.quaternion_from_euler(0, 0, 0), rospy.Time.now(), "adr", '/zed_left_camera_frame') #if using ZED Mini
\end{minted}

The above code example is for calculating the real-word location of $adr$. The output reading of $pc2.read\_points$ is based on camera frame and camera axes. However, in order to move YuMi's arm around this location in later stages, the reading needs to be related to YuMi frame $yumi\_base\_link$ and using YuMi axes format. For ZED Mini, its axes are the same as YuMi's, so the reading format does not need to be modified (see line 5). While for ASUS Xtion, they are different and the corresponding adjustment is shown in line 4 of the code. 

As mentioned in Analysis and Design Chapter, $TF$ keeps the relationship between frames and will used for the frame conversion. The location $adr$ is firstly published using $sendTransdorm$ function, as a transform from camera frame $camera\_link$ or $zed\_left\_camera\_frame$ (see line 6 and 7). Then function $lookupTransform$ can then be used to obtained the location based on frame $yumi\_base\_link$, which will be further discussed in next Section \ref{adj} together with detailed adjustment implementation. Figure \ref{3dadj} shows these frames visualized in Rviz.

\begin{figure}[H]
\centering
\subfigure[Real-world shoe placement]{\includegraphics[height=7cm,keepaspectratio]{Implementation/cv/3dadj.jpg}}
\subfigure[The frame relationships in Rviz]{\includegraphics[height=7cm,keepaspectratio]{Implementation/cv/adjrviz.png}}
\caption{Example of required locations for shoe pose adjustment}
\label{3dadj}
\end{figure}

After adjustment, the direction of the shoe and the interested shoe hole is considered to be toward the side of the camera. Figure \ref{rangeshoe} displays the ideal range of orientations for the shoe hole after adjustment, where the red arrow represents its current orientation. In the rest of this Chapter, the orientation of the shoe hole defaults in this range.

\begin{figure}[H]
\centering
\includegraphics[width = 0.5\columnwidth]{Implementation/cv/rangetoward.jpg}
\caption{Ideal range of orientations for the shoe hole after adjustment}
\label{rangeshoe}
\end{figure}


\section{3D Location of Shoe Hole} \label{3dlocationestimation}
Since the shoe is marked, shoe hole detection can be achieved with color detection. Due to the fact that there are other objects on the workbench and color tracking is very sensitive to light condition, perform this technique using entire image cannot provide accurate results. Therefore, the image will firstly be cropped using $xmin$, $xmax$, $ymin$, and $ymax$ value computed in section \ref{shoedetection}. Color detection only applies to the cropped region of interest.

The image of shoe always contains a certain level of high frequency noise. Various blurring and smoothing techniques have been tried to tackle this issue and their results are shown in Figure \ref{zedfilter} second line. The blurred image is then converted to HSV color space. After that, a mask for color 'blue' is constructed using its predefined lower and color boundaries. The $erode$ and $dilate$ functions are then performed to remove any small blobs remain in the mask. It can be discovered that basically all the four filter can provide a reasonable mask of shoe hole area.

\begin{figure}[H]
\centering
\subfigure{\includegraphics[width = 0.24\columnwidth]{Implementation/cv/zbi1.png}}
\subfigure{\includegraphics[width = 0.24\columnwidth]{Implementation/cv/zg1.png}}
\subfigure{\includegraphics[width = 0.24\columnwidth]{Implementation/cv/zm1.png}}
\subfigure{\includegraphics[width = 0.24\columnwidth]{Implementation/cv/zb1.png}}

\subfigure{\includegraphics[width = 0.24\columnwidth]{Implementation/cv/zbi2.png}}
\subfigure{\includegraphics[width = 0.24\columnwidth]{Implementation/cv/zg2.png}}
\subfigure{\includegraphics[width = 0.24\columnwidth]{Implementation/cv/zm2.png}}
\subfigure{\includegraphics[width = 0.24\columnwidth]{Implementation/cv/zb2.png}}

\subfigure{\includegraphics[width = 0.24\columnwidth]{Implementation/cv/zbi3.png}}
\subfigure{\includegraphics[width = 0.24\columnwidth]{Implementation/cv/zg3.png}}
\subfigure{\includegraphics[width = 0.24\columnwidth]{Implementation/cv/zm3.png}}
\subfigure{\includegraphics[width = 0.24\columnwidth]{Implementation/cv/zb3.png}}

\subfigure{\includegraphics[width = 0.24\columnwidth]{Implementation/cv/zbi4.png}}
\subfigure{\includegraphics[width = 0.24\columnwidth]{Implementation/cv/zg4.png}}
\subfigure{\includegraphics[width = 0.24\columnwidth]{Implementation/cv/zm4.png}}
\subfigure{\includegraphics[width = 0.24\columnwidth]{Implementation/cv/zb4.png}}

\subfigure{\includegraphics[width = 0.24\columnwidth]{Implementation/cv/zbi5.png}}
\subfigure{\includegraphics[width = 0.24\columnwidth]{Implementation/cv/zg5.png}}
\subfigure{\includegraphics[width = 0.24\columnwidth]{Implementation/cv/zm5.png}}
\subfigure{\includegraphics[width = 0.24\columnwidth]{Implementation/cv/zb5.png}}

\subfigure[Bilateral filter]{\includegraphics[width = 0.24\columnwidth]{Implementation/cv/zbi6.png}}
\subfigure[Gaussian filter]{\includegraphics[width = 0.24\columnwidth]{Implementation/cv/zg6.png}}
\subfigure[Median filter]{\includegraphics[width = 0.24\columnwidth]{Implementation/cv/zm6.png}}
\subfigure[Normalized Box filter]{\includegraphics[width = 0.24\columnwidth]{Implementation/cv/zb6.png}}
\caption{Image processing of extracted shoe region using ZED Mini. Each column uses one specific type of blurred filter. For each column, from top to bottom, each image is, in turn, the original image labeled with detected contour area and its centroid coordinates, smoothed image after applying the linear filter, image of HSV color space of the smoothed image, mask for color 'blue', mask after $erode$ function, and the final mask after $dilate$ function}
\label{zedfilter}
\end{figure}

The contour(s) of blue area(s) are then calculated. If at least one contour is discovered, the largest one in the mask will be used and its centroid is computed accordingly. Noticed that this centroid coordinate is for the cropped image. Therefore, $xmin$ and $ymin$ must be added back to give the pixel coordinate of the original image. Finally, if the contour area lies between a predefined range, the calculated centroid will be regarded as the centroid of the shoe hole.

\begin{figure}[H]
\centering
\subfigure[Real-world shoe placement]{\includegraphics[height=7cm,keepaspectratio]{Implementation/cv/3dposw.jpg}}
\subfigure[The frame relationships in Rviz]{\includegraphics[height=7cm,keepaspectratio]{Implementation/cv/3dpos.jpg} \label{3dpos}}
\caption{Example of shoe hole 3D location estimation using ZED Mini}
\end{figure}

Once the 2D pixel position is computed, same approach as in Section \ref{shoeadjust} will be used to obtain its 3D real-world location $shoe\_hole$ referenced to camera frame. Related code can be found in !!!!!!. To improve its stability, 10 readings of $shoe\_hole$ will be recorded and averaged before publishing to $TF$. Figure \ref{3dpos} is the example visualized in Rviz.

\section{3D Orientation of Shoe Hole} \label{3DOrientationofShoeHole}
Due to the fact that the surrounding area of the shoe hole can be regarded as a plane, as shown in Figure \ref{plane}, its 3D orientation is the same as that of the plane. In addition, the orientation of a plane can be obtained from its plane equation. For a plane $Ax + By + Cz = D$, its normal vector is $n = Ai + Bj + Ck$.

To achieve the shoelace insertion task, a pre-insertion point called $pre\_put$ need to be defined along the normal vector direction. With this, YuMi will move its gripper to this location, then align the gripper with the normal vector, and finally move to $shoe\_hole$. After insertion, it will follow the previous path and return to $pre\_put$.

\begin{figure}[H]
\centering
\includegraphics[width = 0.7\columnwidth]{Implementation/cv/plane.png}
\caption{The surrounding area of the shoe hole. The orientation of the orange plane is approximately same as that of the shoe hole within it.}
\label{plane}
\end{figure}

As mentioned in Background Chapter, by extracting all the surrounding point clouds of the shoe hole, RANSAC can then be applied to estimate the plane parameters A, B, C, and D, while ignoring the outliers. Here, the reasonable surrounding area is the contour area detected in Section \ref{3dlocationestimation}. In other words, it consists of all the white points in the final mask image shown in Figure \ref{zedfilter}. For ASUS Xtion camera, these points need to be converted to YuMi's axis format as before. Taking these points as input, following settings are used for RANSAC. Notice that the number of hypothetical inliers is different for these two cameras due to their different resolutions.

\begin{table}[H]
\centering
\resizebox{\columnwidth}{!}{
\begin{tabular}{||c||c|c|c||}
\hline
Camera & Number of hypothetical inliers & Distance threshold & Max iterations \\ \hline\hline
ASUS Xtion & 20 & 0.01 & 200 \\ \hline
ZED Mini & 60 & 0.01 & 200 \\ \hline
\end{tabular}
}
\caption{RANSAC parameters setting}
\label{ransacsetting}
\end{table}

Again, the output plane parameters are referenced to the camera frame, whose direction points into the shoe. Therefore, location $pre\_put$ can be represented as $(-0.1*A, -0.1*B, -0.1*C)$ referenced to $shoe\_hole$. The minus sign is for converting the normal vector direction to face away from the hole, $0.1$ is used to adjust the length between $pre\_put$ and $shoe\_hole$.

Another issue of this approach is that $pre\_put$ has volatility, especially for low resolution camera like ASUS Xtion. The readings fluctuate every time, which are extremely unstable. To tackle this, I firstly introduced a constraint. Since the orientation of the shoe hole is supposed to toward camera side, the x coordinate of $pre\_put$ must be smaller or equal to that of the $shoe\_hole$ (x-axis is the red line in Figure \ref{xaxis}). Therefore, the algorithm will ignore any calculated vector with unreasonable x coordinates. 

\begin{figure}[H]
\centering
\subfigure[Real-world shoe placement]{\includegraphics[height=7cm,keepaspectratio]{Implementation/cv/3dposwa.jpg} \label{xaxis}}
\subfigure[The frame relationships in Rviz]{\includegraphics[height=7cm,keepaspectratio]{Implementation/cv/3dori.jpg}\label{3dori}}
\caption{Example of shoe hole 3D orientation estimation using ZED Mini}
\end{figure}

To further improve the stability and accuracy of the results, every 50 vector coefficients will be recorded when using ASUS Xtion, every 10 vector coefficients will be documented if using ZED Mini, and further data processing will be performed. The following two techniques have been experimented.

\begin{itemize}
    \item \textbf{K-means Clustering:} K-means can partition these $pre\_put$ into k clusters where each point belongs to the cluster with the nearest mean. The cluster with the most number of points will be treated as the correct one and the cluster mean is the final $pre\_put$s. However, for this approach, the number of cluster need to be defined at first. Different settings give very different results even under same environments. Once the selected cluster is wrong, the resulting point can be ridiculous sometimes, which is unreliable.
    
    \item \textbf{Average:} Considering the majority of calculated points are reliable, averaging these 50 $pre\_put$ should give a reasonable answer. This method has been experimented and proven to be effective.
\end{itemize}

The final $pre\_put$ are then be published to $TF$ topic as a transform from frame $shoe\_hole$. Figure \ref{3dori} illustrates all the frames used up to this stage.

%\section{3D Orientation of Shoelace}
\chapter{Implementation - Motion Planning}

This chapter will cover the implementation methods on motion planning side of this project. The entire code for this part can be seen in !!!!!!!!!!

\section{MoveIt! Interface and Planning Scene Setup}
As mentioned in Background Chapter, the motion planning part will utilize MoveIt!. The $moveit\_commander$ namespace is imported in order to use Python MoveIt! interface. This namespace contains a $RobotCommander$ class, a $MoveGroupCommander$ class, and a $PlanningSceneInterface$ class.

A $RobotCommander$ object is firstly instantiated called $robot$. This object is the outer-level interface to the robot, which allow user to catch YuMi's names of joints, groups, and links and easy access to their properties.

To plan and execute motions on the YuMi, $MoveGroupCommander$ objects need to be instantiated as well. Each object should be one group of joints. In this case, the group can be $left\_arm$, $right\_arm$, and $both\_arms$. All of them are defined with the following settings.

\begin{table}[H]
\centering
\resizebox{\columnwidth}{!}{
\begin{tabular}{||c|c|c|c||}
\hline
Pose reference frame & Allow re-planning & Goal position tolerance & Goal orientation tolerance \\ \hline\hline
$yumi\_base\_link$ & False & 0.005 & 0.005 \\ \hline
\end{tabular}}
\caption{Settings of three $MoveGroupCommander$ objects}
\label{armsetup}
\end{table}

Moreover, a $PlanningSceneInterface$ object is instantiated, which is an interface to the world surrounding the robot. In this case, I defined the 3D position and 3D dimension of the workbench in front of YuMi, as shown in Figure \ref{scene}. By doing this, the workbench will be regarded as an obstacle to YuMi, so that the potential collisions between YuMi's arm(s) and the workbench can be avoided. However, the external cameras are not defined due to high measurement difficulty, by setting some safe positions for YuMi, the collisions can also be avoided. The details are discussed in Section \ref{safetyposescalculation}.

\begin{figure}[H]
\centering
\includegraphics[width = 0.5\columnwidth]{Implementation/mp/planningscene.png}
\caption{The planning scene}
\label{scene}
\end{figure}

Finally, a $DisplayTrajectory$ publisher is created, which is utilized to publish trajectories for RViz to visualize:

\begin{minted}[frame=single, framesep=1.5mm, baselinestretch=1, fontsize=\footnotesize, linenos, breaklines]{python}
display_trajectory_publisher = rospy.Publisher('/move_group/display_planned_path', 	moveit_msgs.msg.DisplayTrajectory, queue_size=20)
\end{minted}


\section{Movement Control}
The opening and closing of YuMi' gripper(s) depend on the gripper effort value, which needs to be set between $-20$ to $20$. The gripper will open when the value is negative and close when it is positive. By publishing the effort value of specific gripper to its corresponding topic, $/yumi/gripper\_r\_effort\_cmd$ for right gripper, or $/yumi/gripper\_l\_effort\_cmd$ for left gripper, the gripper control function can be achieved. 

%(Whether the command is to open or close, the effort value will eventually set to 0 in order to relax the effort, which will not change the gripper state.)

As for YuMi's arm(s) control, it can be implemented in two ways: setting a joint goal or setting a pose goal.

\begin{itemize}
    \item \textbf{Joint Goal:} Each of YuMi's arm has 7 joints (shown in Figure \ref{yumijoint}). By setting them to specific value, every joint on the arm will rotate or bend accordingly to that configuration. The position of end-effector can be computed using forward kinematics.
    
    In this project, this approach is only used to adjust YuMi's arm(s) to safe positions through the manipulation process, which will be introduced in Section \ref{safetyposescalculation}.
\end{itemize}

\begin{figure}[H]
\centering
\includegraphics[width = 0.7\columnwidth]{Implementation/mp/yumijoints.png}
\caption{Joints of YuMi's arms \citep{Productspecification}}
\label{yumijoint}
\end{figure}

\begin{itemize}
    \item \textbf{Pose Goal:} The movement of a group can also be controlled by setting a pose goal for YuMi's end-effector(s) directly. The pose contains 3D position (x, y, z) and 3D orientation information (roll, pitch, yaw). The joint parameters that achieve this pose can be calculated via inverse kinematics. Notice that for the same pose goal, the joint parameters can be different.
    
    This method is used throughout the manipulation process including adjusting shoe pose, putting the lace through shoe hole, and grabbing the shoelace etc.
\end{itemize}

For single arm manipulation, letting the end-effector move to a pose goal can be achieved in two ways. Either using $compute\_cartesian\_path$ or $set\_pose\_target$ function. For a given pose goal, the former method will be tried first, and if it fails, the later one will be used. The $compute\_cartesian\_path$ function computes a Cartesian path that follows specified waypoints (6D poses). The maximum resulting step size between the end-effector configurations of consecutive points in the result trajectory is predefined, which is 0.01 that provides a resolution of $1$ cm. Collision and kinematic will be checked and if these constraints cannot be met, the function will return a fraction of the path achieved between $0$ and $1$. If successful, fraction will equal to $1$.

\begin{minted}[frame=single, framesep=1.5mm, baselinestretch=1, fontsize=\footnotesize, linenos, breaklines]{python}
(plan, fraction) = cur_arm.compute_cartesian_path(waypoints, 0.01, 0.0, True)
\end{minted}

While the $set\_pose\_target$ function can be used for both single and dual arm motion planning. Its inputs are simply the target 6D pose and the name of corresponding end-effector. Following is the dual arm plan and move function for this project utilizing $set\_pose\_target$. 

\begin{minted}[frame=single, framesep=1.5mm, baselinestretch=1, fontsize=\footnotesize, linenos, breaklines]{python}
def plan_and_move_dual(target_l, target_r):
    group_both.set_pose_target(target_l, "yumi_link_7_l")
    group_both.set_pose_target(target_r, "yumi_link_7_r")
    plan = group_both.plan()
    group_both.go(wait=True)
    rospy.sleep(3)
\end{minted}



\section{Safe Positions Calculation} \label{safetyposescalculation}

\begin{figure}[H]
\centering
\subfigure[Cal pose]{\includegraphics[width = 0.32\columnwidth]{Implementation/mp/calpose.png}}
\subfigure[Home pose]{\includegraphics[width = 0.32\columnwidth]{Implementation/mp/homepose.png}}
\subfigure[Initial pose]{\includegraphics[width = 0.32\columnwidth]{Implementation/mp/initialpose.png}}
\caption{Safe positions of YuMi}
\label{safeposition}
\end{figure}

Figure \ref{safeposition} shows three pre-defined safe positions of YuMi: cal position, home position and initial position. For each manipulation, YuMi will start with cal position, then move to either home or initial position before perform the tasks. Once it finishes the job, it will move back to home or initial position before finally returning to cal position. By doing so, YuMi's arm(s) will always moving in front of and under the cameras, which prevents potential collisions between them.

After moving YuMi's arm(s) to these desired positions in Rviz, the joint parameters ($safeJointPositionR$ and $safeJointPositionL$) are measured by using $get\_current\_joint\_values$ function. For each safe position, the joint goal(s) can the be set accordingly. Following is the code example to plan and move both arms to a safe position.

\begin{minted}[frame=single, framesep=1.5mm, baselinestretch=1, fontsize=\footnotesize, linenos, breaklines]{python}
elif (arm == BOTH):
    group_both.set_joint_value_target(safeJointPositionL + safeJointPositionR)
    group_both.plan()
    group_both.go(wait=True)
\end{minted}

\section{Shoe Pose Adjustment} \label{adj}
As mentioned in Section \ref{shoeadjust}, if the shoe is vertically placed, then its orientation needs to be adjusted. Recall the idea illustrated in Figure \ref{adjustmentidea}, in which YuMi will use both arms. In Section \ref{shoeadjust}, the real-world locations of $adr$, $adrr$, $adl$, $adll$ have already been computed and published to $TF$.
Here, their locations referenced to YuMi frame $yumi\_base\_link$ can be calculated using $lookupTransform$ function. Following is the example applied on location $adr$.

\begin{minted}[frame=single, framesep=1.5mm, baselinestretch=1, fontsize=\footnotesize, linenos, breaklines]{python}
(trans_adr,_) = self._tfsub.lookupTransform('/yumi_base_link', '/adr', rospy.Time(0))
\end{minted}

The output $trans\_adr$ is the 3D location (x, y, z) of a waypoint. However, the z location and 3D rotation of $adr$ will not be used because they can be pre-defined before the adjustment process. The z coordinate is set as $0.1$ and the rotation is $[-\frac{\pi}{4}, \pi, \pi]$ for left gripper and $[\frac{\pi}{4}, \pi, \pi]$ for right gripper in order to provide the best Angle of application. Figure \ref{realadjust} shows the real adjustment process, where YuMi will start from Cal pose then move to Home pose before goes to (a), and return to Home and finally back to Cal pose after (e). The detailed waypoints for this process can be found in Table \ref{adjustwaypoints}.

\begin{figure}[H]
\centering
\subfigure[Move to pre-adjust]{\includegraphics[width = 0.19\columnwidth]{Implementation/mp/adj2.png}}
\subfigure[Move down to adr and adl]{\includegraphics[width = 0.19\columnwidth]{Implementation/mp/adj3.png}}
\subfigure[Move to adrr and adll]{\includegraphics[width = 0.19\columnwidth]{Implementation/mp/adj4.png}}
\subfigure[Move back to adr and adl]{\includegraphics[width = 0.19\columnwidth]{Implementation/mp/adj5.png}}
\subfigure[Move up to post-adjust]{\includegraphics[width = 0.19\columnwidth]{Implementation/mp/adj6.png}}
\caption{The example motion process of shoe pose adjustment}
\label{realadjust}
\end{figure}

Figure \ref{preposeadjust} shows the detected shoe hole before and after adjustment, the later gives a much more accurate results. Figure \ref{shoeposerange} illustrates the possible orientation range of the shoe hole after adjustment. The hole will face towards the camera, either to hole's top-left or top-right.

\begin{figure}[H]
\centering
\subfigure[Detected shoe hole before and after adjustment]{\includegraphics[height = 7.5cm, keepaspectratio]{Implementation/mp/shoebaadj.png} \label{preposeadjust}}
\subfigure[Ideal orientation range of shoe hole]{\includegraphics[height = 7.5cm, keepaspectratio]{Implementation/mp/toplefttopright.png} \label{shoeposerange}}
\caption{Adjustment results}
\end{figure}


\section{Robot Gripper Approaching Pose}
When the orientation of the shoe is ideal, YuMi will start to put the shoe lace into the hole by using the computed $shoe\_hole$ and $pre\_put$ information. Same as the method used in previous section, these two locations are firstly transformed to $yumi\_base\_link$ frame, which gives the coordinates of (xn, yn, zn) and (x, y, z).

\begin{figure}[H]
\centering
\subfigure[Ideal gripper orientation]{\includegraphics[width = 0.31\columnwidth]{Implementation/mp/ap1.png} \label{gripperap}}
\subfigure[Roll rotation]{\includegraphics[width = 0.31\columnwidth]{Implementation/mp/ap2.png} \label{roll}}
\subfigure[Pitch rotation]{\includegraphics[width = 0.31\columnwidth]{Implementation/mp/ap3.png} \label{pitch}}
\caption{Gripper approaching orientation computation for top-left case}
\end{figure}

Recalling the project requirements, before YuMi starts to manipulate the shoelace, one of its grippers is holding it. In this project, the beginning orientation of the shoelace is considered to be the same as that of the gripper. Therefore, to put it into the shoe hole, the orientation (roll, pitch, yaw) of gripper\slash shoelace needs to be aligned with the normal vector of the plane surrounding with shoe hole. As shown in Figure \ref{gripperap}, for the situation when the shoe hole is orienting to its top-left, this direction is represented as the solid orange line. 

When gripper is facing the ground (the vertical orange dash line in Figure \ref{gripperap}), it has 3D rotation $(0, \pi, \pi)$. Therefore, the alignment can be split into two steps: roll alignment (Figure \ref{roll}) and pitch alignment (Figure \ref{pitch}). There is no need to consider yaw alignment for this step, because the shoelace insertion will not be affected no matter what angle it is. Thus, yaw will be kept as $\pi$.

\begin{equation}
a = \arctan \frac{y - yn}{zn - z}
\label{rollcalculation}
\end{equation}

\begin{equation}
b = \arctan \frac{x - xn}{zn - z}
\label{pitchcalculation}
\end{equation}

By using Equation \ref{rollcalculation} and \ref{pitchcalculation}, the roll and pitch angles can be computed. Because the shoe is assumed in an ideal pose (orienting to its top-left or top-right), $a$ will be in the interval $[-\frac{\pi}{2}, \frac{\pi}{2}]$ and $b$ always lies in range $[0, \frac{\pi}{2}]$. Recall that the beginning configuration of gripper pitch angle is $\pi$, therefore the finally 3D orientation of YuMi gripper should be $(a, b + \pi, \pi)$. 

So far, the preliminary pre-put pose is $[xn, yn, zn, a, b + \pi, \pi]$ and the preliminary insertion pose is $[x, y, z, a, b + \pi, \pi]$. However, these poses are not accurate enough for this precision task, whose offsets need to be further adjusted.

\section{Offset Adjustment}
When letting YuMi's gripper move to a specific 3D location with default orientation $(0, \pi, \pi)$ (facing the ground), the system needs to add $-0.025$ offset on the x-axis ($x\_offset$), $0.005$ offset on the y-axis ($y\_offset$), and $0.145$ offset on the z-axis ($z\_error$) to achieve relatively precise movement. However, the offset adjustment becomes complicate when considering the gripper's orientation.

\begin{figure}[H]
\centering
\includegraphics[width = 0.8\columnwidth]{Implementation/mp/gripperoffset.png}
\caption{YuMi's gripper design, adopted from \citep{Productspecification}}
\label{gripperoffset}
\end{figure}

When planning and moving YuMi's arm(s) by setting pose goals, it is all about the pose of the end-effector. Figure \ref{gripperoffset} shows the design of the YuMi' gripper. It can be discovered that the distance between the endpoint of the gripper and the end-effector is $136mm$. Therefore the offset on z-axis contains $0.136$ gripper offset and $0.009$ system offset ($z\_offset$).

\begin{figure}[H]
\centering
\subfigure[X-axis offset due to pitch rotation]{\includegraphics[width = 0.32\columnwidth]{Implementation/mp/off3.png}\label{offsetx}}
\subfigure[Y-axis offset due to roll rotation]{\includegraphics[width = 0.32\columnwidth]{Implementation/mp/off2.png}\label{offsety}}
\subfigure[Z-axis offset due to roll and pitch rotation]{\includegraphics[width = 0.32\columnwidth]{Implementation/mp/off1.png}\label{offsetz}}
\caption{Gripper offset adjustment for top-left case}
\label{offset}
\end{figure}

Once the gripper performs rotations, this gripper offset will affect the (x, y, z) location the gripper endpoint reaches. In Figure \ref{offset}, the dark blue dash lines represent gripper offset 0.136, the orange dash lines are the corresponding offsets introduced by it. For Figure \ref{offsetx}, if the gripper has a pitch angle of $b$ and the endpoint of the gripper needs to reach the target location, the end-effector only should be at location "End-effector". The same issue occurs for roll rotation as well (see Figure \ref{offsety}). Furthermore, z-axis offset is affected by both of them. To reach the same target location with rotations, the end-effector needs to move down more than that in default orientation. This is because the influence of gripper offset on the z-axis is diluted by the other two axes.

\begin{equation}
xoff = x\_offset - gripper\_offset*sin(b)
\label{xoff}
\end{equation}

\begin{equation}
yoff = y\_offset - gripper\_offset*sin(a)
\label{yoff}
\end{equation}

\begin{equation}
zoff = gripper\_offset*cos(a)*cos(b) + z\_offset
\label{zoff}
\end{equation}

Equation \ref{xoff}, \ref{yoff}, and \ref{zoff} provide the approach to calculate the final offsets. The final pose goals for putting the shoelace into a hole are illustrated in Equation \ref{Preput} and \ref{Insertion}.

\begin{equation}
Preput\_pose = [xn + xoff, yn + yoff, zn + zoff, a, b + \pi, \pi]
\label{Preput}
\end{equation} 

\begin{equation}
Insertion\_pose = [x + xoff, y + yoff, z + zoff, a, b + \pi, \pi]
\label{Insertion}
\end{equation} 

Figure \ref{putexample} shows the insertion process, where YuMi will start from Cal pose then move to (a), and back to Cal pose after finishing the task.

\begin{figure}[H]
\centering
\subfigure[Move to initial pose, and attach shoelace manually]{\includegraphics[width = 0.19\columnwidth]{Implementation/mp/put1.png}}
\subfigure[Move to pre\_put]{\includegraphics[width = 0.19\columnwidth]{Implementation/mp/put2.png}}
\subfigure[Move to shoe\_hole and open the gripper]{\includegraphics[width = 0.19\columnwidth]{Implementation/mp/put3.png}}
\subfigure[Move back to pre\_put]{\includegraphics[width = 0.19\columnwidth]{Implementation/mp/put4.png}}
\subfigure[Move back to initial pose]{\includegraphics[width = 0.19\columnwidth]{Implementation/mp/put5.png}}
\caption{The example motion process of inserting shoelace into a target hole}
\label{putexample}
\end{figure}

\section{Shoelace Grabbing}
After the shoelace being passed through the hole, YuMi should locate this shoelace and pull it out to complete this single hole. Since the endpoint of a shoelace is straight, the orientation of the shoelace is considered to be aligned with this direction. Therefore, the approximately location of shoelace can be computed using the normal vector calculated in Section \ref{3DOrientationofShoeHole}. Given the normal vector $n = Ai + Bj + Ck$, the grab location is set as $(0.05*A, 0.05*B, 0.05*C)$ referenced to $shoe\_hole$, where $0.05$ is used to define the distance. Figure \ref{pick} shows the related frames in Rviz, where $pre\_put$, $shoe\_hole$, and $pick$ are on the same normal vector. Here, the $pick$ will be transformed to $yumi\_base\_link$ frame as well, which gives the coordinate $(xp, yp, zp)$.

\begin{figure}[H]
\centering
\subfigure[Real-world shoe placement]{\includegraphics[height=6cm,keepaspectratio]{Implementation/cv/3dposw.png}}
\subfigure[The frame relationships in Rviz]{\includegraphics[height=6cm,keepaspectratio]{Implementation/cv/pickrviz.png}\label{pick}}
\caption{Example of shoe lace pose estimation using ZED Mini.}
\end{figure}

The idea to solve this problem is shown in Figure \ref{lacegrab}, where the orange arrow represents the direction in which the shoelace is inserted into the hole.

\begin{figure}[H]
\centering
\includegraphics[width = 0.5\columnwidth]{Implementation/mp/lacegrab.png}
\caption{2D Top view that contains the shoe hole position, ideal shoelace pickup position, etc. for top-left case}
\label{lacegrab}
\end{figure}

Therefore, when YuMi's gripper is perpendicular to it, YuMi will have the highest probability to complete this task. In other words, the two fingers of the gripper should be at the two orange points respectively, which aligns with orange dash line. In this case, when the gripper closes, the shoelace can be grabbed. 

The parameter that controls the rotation of the gripper on the z-axis is yaw. When YuMi's fingers are at two blue points in Figure \ref{lacegrab}, the yaw rotation is $\frac{\pi}{2}$. Thus, there is an angle $c$ needs to be added, which can be calculated according to Equation \ref{yawcalculation}.

\begin{equation}
c = \arctan \frac{yp - y}{xp - x}
\label{yawcalculation}
\end{equation}

Considering both top-left and top-right cases, the available range of $c$ is $[-\frac{\pi}{2}, \frac{\pi}{2}]$. When this condition is satisfied, the final grabbing pose can be written as Equation \ref{grabpose}. Notice that, here, the offset for 3D location is different form that of $Preput\_pose$ and $Insertion\_pose$. This is because this time the gripper does not perform any roll and pitch rotations from its default orientation.

\begin{equation}
Grabbing\_pose = [xp + x\_offset, yp + y\_offset, zp + z\_error, 0, \pi, c + \frac{\pi}{2} - 0.1]
\label{grabpose}
\end{equation}

In addition, $-0.1$ is added to yaw rotation. This angle adjustment prevents gripper from getting stuck on the sides of the shoe. The pre-grabbing pose is same as the grabbing pose except its $z$ value is a $0.08m$ higher, which allows the gripper to approach the shoelace more safely.

\begin{figure}[H]
\centering
\subfigure[Move to home pose]{\includegraphics[width = 0.19\columnwidth]{Implementation/mp/pull1.png}}
\subfigure[Move to pre-grabbing and open the gripper]{\includegraphics[width = 0.19\columnwidth]{Implementation/mp/pull2.png}}
\subfigure[Move to grabbing\_pose and close the gripper]{\includegraphics[width = 0.19\columnwidth]{Implementation/mp/pull3.png}}
\subfigure[Move back to pre-grabbing]{\includegraphics[width = 0.19\columnwidth]{Implementation/mp/pull4.png}}
\subfigure[Move back to initial pose]{\includegraphics[width = 0.19\columnwidth]{Implementation/mp/pull5.png}}
\caption{The example motion process of pulling shoelace out}
\label{pickexample}
\end{figure}

Figure \ref{pickexample} displays the grasp process, where YuMi will use its left arm. It starts from Cal pose, then moves to (a) Home pose before executing. After finishing the task, it will return to Initial pose (e) and finally back to Cal pose.
\chapter{Testing and Results}

In order to examine the performance and robustness of my approach, a serious of experiments target both individual functions and the integrated system has been conducted. The success rate and execution times are the two evaluation criteria.

In all experiments, I utilize the same platform as shown in Figure \ref{5.1}. It consists of a YuMi robot and an external camera (ZED Mini or ASUS Xtion). The marked shoe will be placed on the workbench within the area YuMi can reach. All tests were conducted under normal lighting conditions. 

\section{Shoe Detection and Pose Estimation}
The purpose of this part experiment is to assess the robustness of object detection and if the system can correctly compute adjustment locations or the centroid of the shoe hole.

The marked shoe will be placed at random location on the workbench for 10 times. For each location, I rotate the shoe 5 times to adjust its orientation. YOLO algorithm will be used to perform shoe detection and return the prediction probability and the bounding box. Its performance is illustrated in Table \ref{yolotest}.

\begin{table}[H]
\centering
\resizebox{\columnwidth}{!}{
\begin{tabular}{||c||c|c|c|c|c||}
\hline
 & \begin{tabular}[c]{@{}c@{}}Average \\ execution time\end{tabular} & Success rate & \begin{tabular}[c]{@{}c@{}}Average\\ probability\end{tabular} & \begin{tabular}[c]{@{}c@{}}Minimum \\ probability\end{tabular} & \begin{tabular}[c]{@{}c@{}}Maximum \\ probability\end{tabular} \\ \hline \hline
Shoe detection & ms & /50 & 0.7 & 0.41 & 0.95 \\ \hline
\end{tabular}}
\caption{}
\label{yolotest}
\end{table}

Once YOLO detects a shoe, the following two situations will occur. When the shoe is vertically placed, the algorithm should publish the required four adjustment locations ($adr$, $adrr$, $adl$, and $adll$) to $TF$. If the shoe is in a good orientation, the 2D pixel coordinate of centroid of the shoe hole is supposed to be reported.

\begin{table}[H]
\centering
\resizebox{\columnwidth}{!}{
\begin{tabular}{||c||c|c|c||}
\hline
 & Average execution time & Success rate & Maximum error \\ \hline \hline
Adjustment locations & ms & /20 & -- \\ \hline
Centroid coordinate & ms & /30 &  \\ \hline
\end{tabular}}
\caption{}
\label{locationtest}
\end{table}


\section{Shoe Pose Adjustment}
give locations -> if successfully adjust, any collisions?

\section{Insertion and Pulling of The Shoelace}

keep centroid , -> if insert successfully (offset + orientation) -> if pull out

\section{Integrated System Performance}

all of them 

\chapter{Conclusion}

This project successfully addressed the precise AI planning and control problem - putting a shoelace on a shoe using bimanual robot, YuMi, and an external camera. 

Recall Figure \ref{workflow}, the system workflow, the core project which is passing the shoelace into a hole (with $4mm$ radius) has been successfully completed. In addition, YuMi equips with capabilities to adjust the pose of the shoe if necessary, as well as grab and pull the shoelace out to proper location for further manipulation after insertion. The corresponding success rates can be found in Table \ref{sris}.

\begin{table}[H]
\centering
\begin{tabular}{||c|c|c|c||}
\hline
Detection & Adjustment & Insertion & Pulling \\ \hline\hline
100\% & 95\% & 87\% & 91\% \\ \hline
\end{tabular}
\caption{The success rate of integrated system}
\label{sris}
\end{table}

For computer vision part, all the requirements were met. The two cameras, ASUS Xtion and ZED Mini, were correctly set up and calibrated. All computer vision algorithms in this project were implemented for both of them. As discussed in Section \ref{shoedetection}, the shoe can be successfully detected by employing YOLO, regardless of whether there are other interfering objects on the workbench. The calculation of required locations for shoe pose adjustment is introduced in Section \ref{shoeadjust}. The location conversion between camera frame and YuMi frame is based on ROS $TF$ topic. Real-time 6D pose (3D location and 3D orientation) of shoe hole can also be estimated with high precision. The location is calculated based on some image processing techniques including color detection, blurred filter etc, details can be found in Section \ref{3dlocationestimation}. While the method to compute accurate orientation of the shoe hole is mentioned in Section \ref{3DOrientationofShoeHole}, which is based on RANSAC and pre-defined constraints.

The advantages of my vision approaches are fast, relatively stable, and only require one camera. Both the normal vector of the surrounding plane of the shoe hole and color detection have great volatility in the beginning. By extracting ...

For motion planning part, every functionality has been fulfilled as well. My algorithms can successfully plan and move YuMi for shoe and shoelace manipulation while avoiding potential collisions with the environments. The important details about interface and planning scene setup are told in Section \ref{motionplansetup}. The movement control, and related implementation methods such as setting joint goal and pose goals are discussed in Section \ref{movementcontrol}. The three safety poses throughout manipulation process are defined in Section \ref{safetyposescalculation}. Detailed shoe pose adjustment plan, shoelace insertion plan, shoelace grabbing plan and a series of their movement photo examples can be found in Section \ref{adj}, \ref{approachposegripper}, and \ref{shoelacegrabbing} respectively. The accurate offset adjustment approach is in Section \ref{offsetadjustment}.


Back to the workflow, Figure \ref{workflow}, the last two parts: "compute the orientation of the shoelace" and "align the gripper with the shoelace and reclamp" have not been implemented yet. However, the methods are no different from what have been described in the report. The former can be achieved by using color detection and RANSAC. According to the approaches introduced in Section \ref{approachposegripper}, YuMi can align its right gripper with the colored head of shoelace and accomplish reclamping. Adjusting shoe pose using single arm can refer to Section \ref{adj}. After that, the whole workflow is completed and the shoelace can be passed into every hole of the shoe.



%critical evaluation compare to previous ....product, algorithm, ...

 %- design choice 
 %- how overcome 
 %- what did I learn

%advantages (positively and worthwhile) achievement!!

%disadvantages (limit)

\chapter{Future Work}

\section{Complete the Workflow}

\section{Shoe Model Construction}

\section{Deformable Object Tracking}

\section{Gripper or External Wrist Camera}
error feadback

\section{Reinforcement Learning}
\chapter{User Guide}
%put in here maybe pseudocode, or configuration file specifications etc.
This guide aims to support anyone who would like to use the delivered system or continue this project. Before launching this project, following packages must be installed: \texttt{ROS Kinetic}, \texttt{openni}, \texttt{zed-ros-wrapper}, \texttt{python-pcl}, \texttt{darknet\_ros}, \texttt{YuMi-ros-wrapper}, and \texttt{MoveIt!}. Some packages have their own dependencies. The detailed installation instructions can be found in the project Github repository \url{https://github.com/Martinpanzy/FYP-Yumi}.

The pose relationships between the two cameras and YuMi are defined in the YuMi launch files. The \texttt{demo.launch} is for simulation in Rviz, \texttt{yumi.launch} is for launching the real robot. 

For ASUS Xtion camera, the current relationship is defined as:
\begin{minted}[frame=single, framesep=1.5mm, baselinestretch=1, fontsize=\footnotesize, linenos, breaklines]{xml}
<node pkg="tf" type="static_transform_publisher" name="rgb_tf" required="true" args="0.2013 0.0641 0.6934 -0.0238 0.4836 -0.0141 0.875 /yumi_base_link /camera_link 100"/>
\end{minted}

For ASUS Xtion camera, the current relationship is defined as:
\begin{minted}[frame=single, framesep=1.5mm, baselinestretch=1, fontsize=\footnotesize, linenos, breaklines]{xml}
<node pkg="tf" type="static_transform_publisher" name="rgb_tf" required="true" args="0.2263 0.0541 0.7134 -0.0238 0.4836 -0.0141 0.875 /yumi_base_link /zed_left_camera_frame 100"/>
\end{minted}

The resolution and other settings of ZED Mini camera can be modified in the \texttt{common.yaml} file in the \texttt{zed-ros-wrapper} package. For YOLO detection, different versions can be used by modifying the \texttt{yolo.yaml} file in the \texttt{config} folder of \texttt{darknet\_ros} package and adding corresponding \texttt{.cfg} and \texttt{.weights} files to \texttt{yolo\_network\_config} folder. For more detail, please see the Github repository.

The computer vision and motion planning implementation methods introduced in this report are included in folder \texttt{shoehole} and \texttt{shoelace} respectively. Within these two folders:

\begin{itemize}
    \item \texttt{shoe\_asus.py:} provides functions of shoe detection, calculation of required locations for shoe pose adjustment, 6D shoe hole pose estimation etc using ASUS Xtion camera.
    \item \texttt{shoe\_zed.py:} provides same functionalities as shoe\_asus.py except it is for ZED Mini camera.
    
    \item \texttt{go\_asus.py:} provides shoe pose adjustment, shoelace insertion, grabbing, and pulling, as well as offset adjustment functionalities while using ASUS Xtion camera.
    \item \texttt{go\_zed.py:} provides same function as go\_asus.py except it is for ZED Mini camera.
    \item \texttt{yumi\_moveit\_utils.py:} includes several functions which are used in go\_asus.py and go\_zed.py.
\end{itemize}

To launch this project, YuMi needs to be set up firstly. To do this, please follow these steps:
\begin{enumerate}
\item Turn on YuMi through the power switch.
\item Connect the ethernet cable on XP23 port to your machine.
\item Turn on motors on controller interface (toggle physical button with 3 horizontal lines on the FlexPendant).
\item Switch to auto mode (toggle physical button with 2 horizontal lines on the FlexPendant).
\item Point to main programs (toggle physical button with 1 horizontal line on the FlexPendant).
\item Press the play button, both YuMi's grippers should then calibrate.
\end{enumerate}

Once YuMi has been setup, the shoe can be placed on the workbench. Whne using ZED Mini camera, following command should be run from a terminal after \texttt{roscore}:
\begin{enumerate}
\item \texttt{roslaunch zed\_wrapper zedm.launch}: to launch ZED Mini camera
\item \texttt{roslaunch darknet\_ros darknet\_ros.launch}: to launch YOLO detection
\item \texttt{roslaunch yumi\_shoe yumi.launch}: to launch YuMi ROS Nodes
\item \texttt{rosrun yumi\_shoe shoe\_zed.py}: to launch computer vision module
\item \texttt{rosrun yumi\_shoe go\_zed.py}: to launch motion planning module
\end{enumerate}

YuMi will then starts to manipulate the shoe or shoelace depends on the pose of the shoe. For more information, please see the project Github repository.


\bibliographystyle{plainnat}
\bibliography{bibs/sample}
\appendix
\chapter{Figures and Tables}
%Information typically included are things like program listings, complex circuit diagrams, tables, proofs, graphs or any other material which would break up the theme of the text.

This appendix presents several examples of photos and results when examining the computer vision techniques introduced in Implementation - Computer Vision Chapter using ASUS Xtion camera. In addition, some detailed waypoint settings mentioned in Implementation - Motion Planning Chapter are included as well.

\begin{figure}[H]
\centering
\subfigure[RGB raw image]{\includegraphics[width = 0.45\columnwidth]{Implementation/cv/raw.png}}
\subfigure[YOLO detection image including bounding boxes]{\includegraphics[width = 0.45\columnwidth]{Implementation/cv/yolo.png}}
\caption{ASUS Xtion camera image before and after using YOLO detection}
\label{5.2asus}
\end{figure}

\begin{figure}[H]
\centering
\subfigure[RGB raw image]{\includegraphics[height=5cm,keepaspectratio]{Implementation/cv/rgbpt.png}}
\subfigure[Depth registered point clouds]{\includegraphics[height=5cm,keepaspectratio]{Implementation/cv/ptcloud.png} \label{ptcloud}}
\caption{Messages of different ASUS Xtion camera topics}
\end{figure}

\begin{figure}[H]
\centering
\subfigure{\includegraphics[width = 0.24\columnwidth]{Implementation/cv/bi1.png}}
\subfigure{\includegraphics[width = 0.24\columnwidth]{Implementation/cv/g1.png}}
\subfigure{\includegraphics[width = 0.24\columnwidth]{Implementation/cv/m1.png}}
\subfigure{\includegraphics[width = 0.24\columnwidth]{Implementation/cv/b1.png}}

\subfigure{\includegraphics[width = 0.24\columnwidth]{Implementation/cv/bi2.png}}
\subfigure{\includegraphics[width = 0.24\columnwidth]{Implementation/cv/g2.png}}
\subfigure{\includegraphics[width = 0.24\columnwidth]{Implementation/cv/m2.png}}
\subfigure{\includegraphics[width = 0.24\columnwidth]{Implementation/cv/b2.png}}

\subfigure{\includegraphics[width = 0.24\columnwidth]{Implementation/cv/bi3.png}}
\subfigure{\includegraphics[width = 0.24\columnwidth]{Implementation/cv/g3.png}}
\subfigure{\includegraphics[width = 0.24\columnwidth]{Implementation/cv/m3.png}}
\subfigure{\includegraphics[width = 0.24\columnwidth]{Implementation/cv/b3.png}}

\subfigure{\includegraphics[width = 0.24\columnwidth]{Implementation/cv/bi4.png}}
\subfigure{\includegraphics[width = 0.24\columnwidth]{Implementation/cv/g4.png}}
\subfigure{\includegraphics[width = 0.24\columnwidth]{Implementation/cv/m4.png}}
\subfigure{\includegraphics[width = 0.24\columnwidth]{Implementation/cv/b4.png}}

\subfigure{\includegraphics[width = 0.24\columnwidth]{Implementation/cv/bi5.png}}
\subfigure{\includegraphics[width = 0.24\columnwidth]{Implementation/cv/g5.png}}
\subfigure{\includegraphics[width = 0.24\columnwidth]{Implementation/cv/m5.png}}
\subfigure{\includegraphics[width = 0.24\columnwidth]{Implementation/cv/b5.png}}

\subfigure[Bilateral filter]{\includegraphics[width = 0.24\columnwidth]{Implementation/cv/bi6.png}}
\subfigure[Gaussian filter]{\includegraphics[width = 0.24\columnwidth]{Implementation/cv/g6.png}}
\subfigure[Median filter]{\includegraphics[width = 0.24\columnwidth]{Implementation/cv/m6.png}}
\subfigure[Normalized Box filter]{\includegraphics[width = 0.24\columnwidth]{Implementation/cv/b6.png}}
\caption{Image processing of extracted shoe region using ASUS Xtion camera. Each column uses one specific type of blurred filter. For each column, from top to bottom, each image is, in turn, the original cropped image labeled with detected contour area and its centroid coordinate, smoothed image after applying the linear filter, the smoothed image in HSV color space, mask for color 'blue', mask after $erode$ function, and the final mask after $dilate$ function.}
\label{asusfilter}
\end{figure}

\begin{table}[H]
\centering
\resizebox{\columnwidth}{!}{
\begin{tabular}{||c||c|c||}
\hline
Waypoints & Left arm & Right arm \\ \hline\hline
$(a)$ & $[adl\_x, adl\_y, 0.3, -\frac{\pi}{4}, \pi, \pi]$ & $[adr\_x, adr\_y, 0.3, \frac{\pi}{4}, \pi, \pi]$ \\ \hline
$(b)$ & $[adl\_x, adl\_y, 0.1, -\frac{\pi}{4}, \pi, \pi]$ & $[adr\_x, adr\_y, 0.1, \frac{\pi}{4}, \pi, \pi]$ \\ \hline
$(c)$ & $[adll\_x, adll\_y, 0.1, -\frac{\pi}{4}, \pi, \pi]$ & $[adrr\_x, adrr\_y, 0.1, \frac{\pi}{4}, \pi, \pi]$ \\ \hline
$(d)$ & $[adl\_x, adl\_y, 0.1, -\frac{\pi}{4}, \pi, \pi]$ & $[adr\_x, adr\_y, 0.1, \frac{\pi}{4}, \pi, \pi]$ \\ \hline
$(e)$ & $[adl\_x, adl\_y, 0.3, -\frac{\pi}{4}, \pi, \pi]$ & $[adr\_x, adr\_y, 0.3, \frac{\pi}{4}, \pi, \pi]$ \\ \hline
\end{tabular}}
\caption{Shoe pose adjustment waypoints for both of YuMi's end-effectors}
\label{adjustwaypoints}
\end{table}

\end{document}